\chapter{Methodology}

\section{Introduction}
It must be made clear that due to its complexity and scope, the design of this project was not an individual work, but a team of two people worked on it. The team leader Javier Roldan is an experienced telecommunications engineer who works as a satellite ground station engineer and LEOP ground operations manager for ESA/ESOC. Together we worked on the backup voice communication system for the Serenity spacesuit simulator. Our OeWF supervisor was Michael Müller who is an experienced engineer and physicist. All decisions regarding this project were discussed and made together.

Even though the design of the backup system was made by both, Javier and me, all calculations and simulations were conducted by me. Because calculation and simulation results have to be discussed, Javier gave me good advices on how I can improve my calculations and simulations along the way.

Now that the role of the team members is clear, the following sections will formulate the research question, list the delivarebles for this project and explain how this project was scientifically approached.

\section{Research question}


\section{Deliverables}
The deliverables for this project are summarised in the following list.

\begin{itemize}
	\item Design of the backup voice communication system for the Serenity spacesuit simulator.
	\item Construction of the designed backup voice communication system.
	\item Verification that the constructed voice communication system meets the system requirements mentioned in table \ref{tab:table_serenity_bu_comms_requirements}.
	\item Providing a MATLAB program to simulate the energy consumption of the self-sufficient backup voice communication repeater (BVCR).
	\item Evaluation of a planetary voice communication system on Mars based on Serenity's backup system.
\end{itemize}

\section{Workflow}
