\chapter{Infrastructure research}

\section{Introduction}
Due to the fact that the OeWF conducted plenty of successful missions, hardware and documentation for the primary system as well as for the current backup system is available. This is a big advantage because mission procedures and important data, regarding the communication with the AAs during field missions is already known. 

\section{Current hardware}

% FROM INTRODUCTION
For this purpose Aouda used Motorola Mototrbo DP 3601 portable radios, which operate in the very high frequency (VHF) band and support analog and digital modulation. But these were only the first version and they were quickly discarded in the testing phase because their high effective radiated power (ERP) of $P_{TX, Low} = 1\mathrm{W}$ to $P_{TX, High} = 5\mathrm{W}$ while transmitting, interfered with the $\mathrm{I^2C}$ data bus of the on-board data handling (OBDH) system. For this reason less powerful Kenwood ProTalk Digital TK-3401D digital personal mobile radios (dPMRs), which operate on the free PMR446 license and only have an ERP of $P_{TX} = 0,5\mathrm{W}$ while transmitting, were chosen. With these dPMRs the OeWF achived a decent range of $\mathrm{6,5km}$ while maintaining direct line of sight (LOS) \cite{DP3601:2010, Kenwood:2014, Groemer:2020}. 
% END FROM INTRODUCTION

As already mentioned in the previous chapter, Aouda currently uses a primary system which is based on a $f = 5\mathrm{GHz}$ WLAN for the main communication pathway between the astronaut and the mission crew, including a base camp infrastructure and repeater stations (Access-Points) in the field, and another $f = 2,4\mathrm{GHz}$ WLAN network within the suit for easy component communication. Its backup system is based upon industrial Kenwood ProTalk Digital TK-3401D dPMRs, which from now on will simply be called Kenwood dPMRs, and Motorola Mototrbo DP3601 handheld radios, which from now on will simply be called DP 3601s. The safety officers currently use the DP 3601s to communicate with each other and the base camp and the analog astronauts use Kenwood dPMRs, which are built into their suit, with analog modulation to communicate with another Kenwood dPMR mounted to a mast at the base camp. Analog modulation was chosen by the OeWF they rather have a bad voice quality than no communication at all in case of bad signal reception. To be able to communicate with the Kenwood dPMR on the mast, the BACA crew uses the Mumble chat application, which was chosen by the OeWF because of its low latency and high voice quality \cite{Mumble:2020}.

After this short summary of the current voice communication system for the Aouda suit, two main questions regarding the design of Serenity's voice communication system, arise. The first one is whether the current backup system infrastructure should be reused for the new backup system or whether new technologies must be procured. And what the mass of the current radios is and what impact this has on the suits's mass budget of $m_{BDG} = 1.000,00\mathrm{g}$.

\subsection{Motorola Mototrbo 3601 handheld radio}
Before these radios can be reused, they first have to be examined in an anechoic chamber to verify if they are still functional. For this purpose, two DP 3601 radios, with the radio numbers 00000011 and 00000012, were sent from the OeWF to the Vienna University of Technology (TUW). In the first step, the basic functionality was checked by simply charging them and reading their radio status, which can be seen in table \ref{tab:table_dp3601_current_programming}, and in the second step it was checked whether both radios can receive and send voice communication. Finally the radios were connected to a personal computer (PC) running the Motorola Mototrbo CPS to study their programming. These results can also be seen in table \ref{tab:table_dp3601_current_programming}.

\vfill%
\begin{table}[h!]
	\centering
	\begin{tabular}{|l|r|}
	\hline
	\multicolumn{2}{|c|}{\textbf{Existing Motorola Mototrbo DP 3601 information}} \\
	\hline
	Firmware Version & R10.09.10 \\
	CPS Version & 11.00.14 \\
	Battery first used & 28.01.2013 \\
	Battery service life & 100\% \\
	\hline
	\multicolumn{2}{|c|}{\textbf{Programmed channels for 158,950MHz}} \\
	\hline
	\textbf{Channel name} & \textbf{Modulation} \\
	\hline
	OeWF 1 & Digital \\
	OeWF 2 & Digital \\
	OeWF Suit 1 & Digital \\
	OeWF Suit 2 & Digital \\
	OeWF SciTech & Digital \\
	OeWF Analog & Analog \\
	\hline
\end{tabular}
	\caption{Information about the existing Motorola Mototrbo DP 3601 VHF handheld radios.}
	\label{tab:table_dp3601_current_programming}
\end{table}
\vfill%

\section{Requirements}
To address the raised questions, the primary system has to be examined. Table \ref{tab:table_primary_comm_system_parts_list} lists the parts of the primary system used in the Serenity suit with the mass of each component \cite{Laird-antenna:2014, Pulse-Larsen-antennas:2015, Laird-cable:2015}. If the amount of a required component is greater than one, the total mass corresponds to the mass of one component times its required amount. For the components MikroTik RBM33G, MikroTik R11e-5HnD, MikroTik R11e-2HnD and Amphenol 082-66-RFX the mass is not specified in their data sheet and was therefore measured in the OeWF's suit laboratory using a classic scale.

\vfill%
\begin{table}[h!]
	\centering
	\begin{tabular}{|c|l|l|r|}
	\hline
	\multicolumn{4}{|c|}{\textbf{Primary COMM system parts list}} \\
	\hline
 	\textbf{Amount} & \textbf{Component} & \textbf{MPN} & \textbf{Total mass} \\
 	\hline
 	2 & $5\mathrm{GHz}$ antenna & PulseLarsen W5030 & $332,00\mathrm{g}$ \\
 	1 & Routerboard & MikroTik RBM33G  & $120,00\mathrm{g}$ \\
 	1 & Dual chain wireless card & MikroTik R11e-5HnD & $18,10\mathrm{g}$ \\%
 	1 & MiniPCIe card & MikroTik R11e-2HnD & $18,10\mathrm{g}$ \\%
 	1 & $2,4\mathrm{GHz}$ antenna & Laird 001-0014 & $1,13\mathrm{g}$ \\
 	2 & Antenna cable & Laird CA100-NM-MMCX-12 & $86,00\mathrm{g}$ \\
	2 & Antenna socket & Amphenol 082-66-RFX & $30,00\mathrm{g}$ \\
 	\hline
\end{tabular}
	\caption{Parts list of the primary voice communication system for the Serenity space suit simulator.}
	\label{tab:table_primary_comm_system_parts_list}
\end{table}
\vfill%

Summed up the mass of the primary system is $m_{PR} = 605,33\mathrm{g}$, which leaves a maximum mass budge for the backup system of $m_{BU, max} = m_{BDG} - m_{PR} = 394,67\mathrm{g}$. Since the primary system for Serenity is not finished yet, a small margin for its missing parts, like a headset, wiring, mechanical assembly parts, etc., has to be considered. 

To get an even better view on the mass budget, the data sheets of the handheld radios for the current backup system have to be examined as well. Table \ref{tab:table_kenwood_specs} lists the specifications of the Kenwood ProTalk Digital TK-3401D dPMR with the KNB-45L Li-Ion battery and table \ref{tab:table_dp3601_specs} lists the specifications of the Motorola Mototrbo DP 3601 VHF handheld radio\footnote{As per Motorola Solutions the average battery life of the DP 3601 applies with the battery saver enabled in carrier squelch and the transmitter in high power} \cite{DP3601:2010, Kenwood:2014, Groemer:2020}.

\vfill%
\begin{table}[h!]
	\centering
	\begin{tabular}{|l|r|}
	\hline
	\multicolumn{2}{|c|}{\textbf{Kenwood ProTalk Digital TK-3401D dPMR specifications}} \\
	\hline
	Number of Channels & 32 channels/2 zones \\
 	Analog Frequency Range & \numrange{446,0}{446,1}MHz \\
 	Digital Frequency Range & \numrange{446,1}{446,2}MHz \\
	Dimensions $\mathrm{(H \times W \times D)}$ & $54,0 \times 122,0 \times 35,5 \mathrm{mm}$ \\
	Weight & 280g \\
	Operating Voltage & 7,5V DC $\pm$ 20\% \\
	Operating Temperature & \numrange{-30}{60}$^\circ \mathrm{C}$ \\
	\hline
	\multicolumn{2}{|c|}{\textbf{Receiver}} \\
	\hline
 	Analog Channel Spacing & 12,5kHz \\
	Digital Channel Spacing & 6,25kHz \\
	\hline
	\multicolumn{2}{|c|}{\textbf{Transmitter}} \\
	\hline
 	RF Power Output & ERP 500mW \\
	Narrow Modulation & 8K50F3E \\
	Very Narrow modulation & 4K00F1E \\
 	\hline
\end{tabular}
	\caption{Specifications for the Kenwood ProTalk Digital TK-3401D dPMR.}
	\label{tab:table_kenwood_specs}
\end{table}
\vfill%

Even though the Kenwood radios have a mass of $m_{KeWo} = 280\mathrm{g}$, which means that, regarding the mass budged, they would theoretically fit inside the suit, they were quickly discarded due to their low ERP of $P_{Tx} = 0,5\mathrm{W}$. As mentioned beforehand the OeWF tested the range of these radios in the field and they concluded,  that no easy to procure repeater infrastructure which would increase the range of the radios \cite{Groemer:2020}. Another downside is the frequency range in which they operate. The free PMR446 license does not require any special training. This can be beneficial to the astronauts as they do not need additional training to use these radios. However, it is also a major disadvantage because anyone can use these radios and therefore anyone can interrupt the backup communication system, which can lead to a life threatening situation for the AAs. To reduce the risk of such a situation to emerge it is clear that the backup system for Serenity needs to operate in a different, less used, frequency range. This led to the decision that the Kenwood radios will not be used in Serenity's backup system, but rather be used by the ground crew at the base camp for everyday, non-critical, communication.

\vfill%
\begin{table}[h!]
	\centering
	\footnotesize
\begin{tabular}{|l|c|}
	\hline
	\multicolumn{2}{|c|}{\textbf{Motorola Mototrbo DP 3601 (VHF digital) specifications}} \\
	\hline
 	Frequency & $136\mathrm{MHz}$ to $174\mathrm{MHz}$ \\
 	Dimensions $\mathrm{(H \times W \times D)}$ & $131,5 \times 63,5 \times 35,2 \mathrm{mm}$ \\%
 	Mass (incl. $2200\mathrm{mAh}$ li-ion battery) & $361\mathrm{g}$ \\%
	Power supply & $7,5\mathrm{VDC}$ (nominal) \\
 	Average battery life at 5/5/90 duty cycle & $19,0\mathrm{h}$ \\
	Operating temperature & $-30^\circ \mathrm{C}$ to $60^\circ \mathrm{C}$ \\
	Storage temperature & $-40^\circ \mathrm{C}$ to $85^\circ \mathrm{C}$ \\
	RX sensitivity 5\% BER: & $0,30\mathrm{\mu V}$ \\
	TX power output & $1\mathrm{W}$ to $5\mathrm{W}$ \\
	IP rating & $57$\\
	\hline
\end{tabular}
	\caption{Specifications for the Motorola Mototrbo DP 3601 VHF handheld radio.}
	\label{tab:table_dp3601_specs}
\end{table}
\vfill%

The situation is very different for the DP 3601. It already operates in the VHF band and it can be set to a low output power of $P_{TX, Low} = 1\mathrm{W}$ or a high output power of $P_{TX, High} = 5\mathrm{W}$. This increases the range of the DP 3601 compared to the Kenwood radios by a factor of 10. Although it looks promising, it has a mass of $m_{\mathrm{DP 3601}} = 361\mathrm{g}$ which makes it too heavy to be used as the internal backup radio. It was decided, that these radios should be somehow embedded into the future backup system for Serenity. This saves costs and it promotes the idea of reusing old infrastructure.

\section{OeWF mission procedures}
