\subsection{Martian application}
In the previous subsections, some important factors were neglected in order to provide a rough performance estimation of the voice communication system on Earth. For an application on Mars these factors should be considered when designing such a system. 

First, the influence of the internal noise of the components -- which is due to their thermal noise -- as well as the influence of the electromagnetic noise collected by the receiving antenna, must be considered. Antennas in the $30\mathrm{MHz}$ to $1\mathrm{GHz}$ range primarily pick up galactic noise which is caused by the radiation produced by electrons moving through the galactic magnetic field of the Milky Way. Due to the shape of the Milky Way and the location of the Earth within it, this type of noise is anisotropic in nature. It increases when the antenna is pointed directly at the center of the Milky Way. For a more detailed examination of the link budget of a voice communication system on Mars, the galactic noise affecting the system would have to be examined more closely. Because of the low temperatures on Mars, the thermal noise of the components and radio devices will probably be lower. However, due to the thinner atmosphere, the heat generated by these cannot be dissipated, which in turn increases the temperature of the system and thus its thermal noise. This problem must be solved with suitable cooling. For completeness it must be mentioned that interference caused by other missions or electrical devices on the surface of Mars or in its orbit should not be neglected when planning a voice communication system. Such interferences can be identified as man-made-noise \cite{Glover:2010, Goiser:2019, Kemmetmuller:2021}. 

Second, a topological map of the mission location can be created, on the basis of which a multipath propagation simulation is carried out so that the radio infrastructure can be optimally placed. In contrast to the voice communication system of the OeWF, which is constantly being set up at new mission locations, it is assumed that the radio infrastructure on Mars will only be set up once and then expanded with further missions. Thus, such an elaborate simulation of the multipath propoagation can be usefull. It is desirable to set up a functioning voice communication network with as little infrastructure as possible in order to save weight and minimize assembly time \cite{Ho:2002}.

Third, the propagation of VHF waves in relation to the Martian atmosphere must be investigated. The authors of \cite{Ho:2002} have found out that Mars has almost no intrinsic magnetic field, which means that the use of the ionosphere for wave propagation is very dependent on the time of day and the season. During the day, the ionosphere can be used as a reflector for global communication, whereas this may not be possible at night. The authors also assumed that the attenuation of a VHF signal due to the ionosphere is approximately $0,5\mathrm{dB}$. The effects of storms in the ionosphere have not yet been researched enough to make any statements about them.

Since the troposphere of Mars is very thin, it is believed that it has little effect on the propagation of electromagnetic waves \cite{Ho:2002}.

With regard to dust storms, the authors of \cite{Ho:2002} further state that due to the large wavelength of VHF signals, compared to the size of Martian dust particles -- with adiameter of $0,1\mathrm{mm}$ -- the signal attenuation is rather small. However, this still needs to be researched, as dust density also plays a major role. 

Fourth, the system must have sufficient redundancy so that a total failure is very unlikely. Loss of communication can potentially be dangerous for a crew during an ongoing mission. 
