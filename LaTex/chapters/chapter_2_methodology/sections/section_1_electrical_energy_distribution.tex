\section{Electrical energy distribution} \label{sec:methodology}
A self-sufficient voice communication system constist of at least one or more electrical consumers that must be supplied with electrical energy. This requires a close examination of the system's electrical energy distribution, which includes an energy source, the energy conversion with a suitable electrical generator, internal electrical energy losses and -- if necessary -- an electrical energy storage device.

The decision to design a self-sufficient voice communication system was made due to a number of reasons. First, it makes the system independent of its location. This of course only applies if the system is operated in a location where there is a source of energy that can be converted by its electrical generator. As a result, it can operate without any external electrical energy source, which offers independence from a local power grid or other on-site electrical generators, such as fuel combustion generators.\footnote{During some of their previous missions the OeWF used diesel generators at the base camp to supply electrical consumers.} 

This already leads to the second and third reason regarding the environmentally harmful way in which electrical energy is generated by burning fossil fuels and the impracticality of fuel combustion generators. With consideration for the planet Earth and its sensitive ecosystem, as well as for future generations, the technologies available today must be used in such a way that the environmental impact of the use of the voice communication system is minimal. Furthermore, looking ahead to future Mars missions, the third reason arises from the fact that its just not a feasable option to bring fuel combustion generators -- and the infrastructure required -- to the Martian surface \cite{Bertol:2011, Mertens:2015}.  

Based on these reasons it was decided to supply the self-sufficient voice communication system from renewable energy sources, such as wind energy, hydroelectric energy, solar energy, bioenergy from biomass, tidal energy etc. Hydroelectric and tidal energy were quickly disregarded because there are no bodies of water on the Martian surface. And since the OeWF conducts Mars analog missions on Earth it is not likely that simulation will take place in locations where there are bodies of water that could provide these sources of energy. Bioenergy from biomass can be disregarded due to similar reasons. 
