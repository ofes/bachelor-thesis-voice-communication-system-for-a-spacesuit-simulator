\chapter{Conclusion and critical reflection}
In addition to the summary and conclusion of the results, a critical examination of this thesis is presented in the following sections. 

\section{Conclusion}
In this section the findings and results of this thesis are summarized and concluded. First the results of the designed voice communication system are presented, followed by the self-sufficient energy distribution system and finally the usability of the system on Mars is discussed. 

\subsection{Voice communication system}
It was shown that a voice communication system can be designed according to requirements of the OeWF and that a performance estimation can be carried out on it. The designed voice communication system consists of a base station radio infrastructure, a radio system for the Serenity spacesuit simulator and a repeater radio infrastructure which is used to increase the covered mission area. It furthermore consists of handheld radios used by the the on-site support crew and the safety officers.

Both the Motorola Mototrbo DP 3601 handheld VHF radios and the existing frequency license for $158,950\mathrm{MHz}$, owned by the OeWF, were reused. In addition to the existing infrastructure, further Motorola Mototrbo radios were procured, with which an infrastructure was designed so that a theoretical mission area with a diameter of $15\mathrm{km}$ can be covered. 

The model for plane earth signal budget was calculated for two cases to show that the fade margins of the voice communication system do not fall below a minimum value of $20\mathrm{dB}$. In the first case it was assumed that all designed radio systems in the voice communication system were transmitting with maximum transmission power and the second case took into account the limitation of the transmission power to a maximum of $5\mathrm{W}$. In both cases the minimum fade margin resulted in $21,595\mathrm{dB}$.

With regard to the Serenity spacesuit simulator, a place for the Serenity radio system in the HUT of the spacesuit simulator was found in such a way that its antenna is installed vertically at a height of $1,65\mathrm{m}$ above the ground for a standard male AA, or at a height of $1,55\mathrm{m}$ above the ground for a standard female AA. The radio system is installed in such a way that it can be recharged in this position and removed if necessary. An approach with a wired PTT button and a wireless Bluetooth headset was chosen to reduce the cabling effort. This also has the advantage that an AA can put on the headset before donning and simply step into the spacesuit simulator. The same applies to doffing. 

The requirement for the operating temperature of the Serenity radio system could not be fully demonstrated. The data sheet of the radio device, the PTT button, the cables and the connections involved show that they meet this requirement. However, no specifications could be found for the Bluetooth headset and the audio jack that connects the PTT button to the radio device. No experiments were carried out either. In order to show that these components meet the temperature requirement, further experiments must be carried out. Alternatively, a Bluetooth headset was presented which fulfills this requirement and is compatible with the radio system. Furthermore, an audio jack that meets this requirement can be requested directly from Motorola Solutions inc. 

Since the W-LAN based voice communication system has not yet been fully developed, no qualitative statement can be made regarding the requirement for the combined mass of the two radio systems. Their combined mass is currently $853,33\mathrm{g}$ with the proposed Bluetooth headset and $866,33\mathrm{g}$ with an alternative Bluetooth headset.  

\subsection{Self-sufficient energy distribution system}
Because the repeater radio system cannot be recharged as easy due to its remote location in the field, an environmentally friendly self-sufficient energy distribution system was designed to supply it with electrical energy. In general it consists of one ore more PV generators connected in parallel, a solar charging controller and a $\mathrm{LiFePO}_4$ battery. A \MATLAB simulation was developed which provides an estimation of wether the self-sufficent energy distribution system can supply the repeater radio infrastructure with enough electrical energy at a given mission location or not. This simulation is based on models of the angular relationships between the Sun and Earth, photovoltaic generators and $\mathrm{LiFePO}_4$ batteries. To obtain a model from the latter, discharging and charging experiments were carried out.

\MATLAB simulations of the self-sufficient energy distribution system for the upcoming analog Mars field mission AMADEE-20 in October 2021 in the Negev Desert in Israel showed that the repeater radio system can be sufficiently supplied with electrical energy from one PV generator. Similar results could be obtained for Vienna, Austria, in April 2021. It was shown, however, that at least two PV generators in parallel are required.

\subsection{Martian application}
When examining the self-sufficient energy distribution system, two components of the system emerged which required special attention. First, the PV generator, as it requires a larger energy converting area due to the lower solar radiation on Mars, and second, the $\mathrm{LiFePO}_4$ battery due to its temperature dependece. An equation was derived which compares the area ratio between a PV generator on Earth and one on Mars. However, this should only be used as a rough estimation, since the temperature dependence of said generator on the surface of Mars has to be investigated more closely. Taking into account dust storms, it was found that during such storms, due to the diffuse component of sunlight, still enough electrical energy can be converted. However, dust that accumulates on the PV generator's energy converting area can become a problem in the long term. Regarding $\mathrm{LiFePO}_4$ batteries, it was found that these have a high temperature dependence and therefore have to be heated or cooled accordingly. For this, a higher energy consumption must be planned.

After examining the voice communication system, it was found that the temperature of the radio devices and components has an influence on their thermal noise. Similar to the self-sufficient energy system, the temperature dependence also needs to be examined in more detail here. So that a reliable voice communication system can be set up on the surface of Mars, it is advisable to carry out a more detailed investigation of the multipath propagation of the electromagnetic waves, so that the infrastructure can be optimally placed. During the day, the Martian ionosphere can be used as a reflector for global communication for frequencies in the VHF range. At night, however, this effect is limited because Mars has almost no intrinsic magnetic field. This effect therefore depends heavily on the time of the day. It is assumed that the Martian troposphere has no influence on the propagation of electromagnetic waves, as it is very thin. The impact of dust storms needs further investigation. However, it is assumed that the signal attenuation for the VHF band is rather low. Finally, sufficient redundancy must be ensured so that a failure of the voice communication system is very unlikely, as this can become dangerous for astronauts. 


\section{Critical refelction}
A critical reflection of this thesis is now presented to point out the areas in which further studies need to be carried out. First the voice commuication system will be examined and then the self-sufficient energy distribution system.

\subsection{Voice communication system}
As already mentioned, a number of factors that influence the performance of the designed voice communication system were neglected. In the literature \cite{Glover:2010} it is explained that special consideration should be given to the temperature of the individual radio devices and components in order to investigate their thermal noise. Independent of this noise source, this literature also deals with other noise sources that could have an influence on the system. Furthermore, a more detailed examination of the signal budget should be carried out in order to take into account possible obstacles in the path of the electromagnetic waves. This includes, for instance, ground roughness. The basis for this can be found in books \cite{Parsons:2000, Mecklenbrauker:2017}. For a Martian application this topic is covered in the NASA publication \cite{Ho:2002}. Subsequently, a simulation with topological maps for different mission locations can be carried out. Finally, the influence of the Earth's atmosphere on the propagation of electromagnetic waves should be taken into account \cite{Glover:2010, Parsons:2000}. In order to obtain empirical data to get a better model of the voice communication system for simulations, experiments need to be carried out with the assembled voice communication system in different locations \cite{LinkMargin:2016}. 

\subsection{Self-sufficient energy distribution system}
Instead of the approximated equations for the angular relationships between the Sun and Earth, databases can be used which contain measured values of these angles throughout the year. Thus, the accuracy of the \MATLAB simulation can be improved. In addition, instead of the Gaussian fitting function -- which was used to simulate the radiation flux $\Phi_\mathrm{G}$ -- a more precise curve shape as shown in \cite{Babikir:2020} can be used. 

Regarding the PV generator, the \emph{two diodes model} can be used to improve the accuracy of the \MATLAB simulation as well. This model is much more accurate because it takes into account the parallel resistance, the series resistance and the leakage currents of a PV cell. However, in order to obtain the necessary parameters for this model, various experiments must be carried out on a PV generator \cite{Mertens:2015, Wagner:2018}. Furthermore, the effect of shade on the PV generator's energy converting area must be investigated, as well as its power output for different weather conditions \cite{Mertens:2015}.

To further increase the accuracy of the \MATLAB simulation, a temperature model \cite{Hausmann:2013, Ala-A.-Hussein:2015, Chin:2018}, a capacity fading model \cite{Li:2018} and the transient (dynamic) behavior, which is described by the Thevenin model \cite{He:2011, Rahmoun:2012, Hentunen:2014, Li:2018, Hinz:2019, Hossain:2019, Saldana:2019}, of the $\mathrm{LiFePO}_4$ battery should be considered.

Finally, the \MATLAB simulation can be sped up by reducing the amount of \codeword{for} loops and by removing possible bugs. Improvements of its accuracy can also be achieved by implementing the described step function of the load current and the charging behavior of the $\mathrm{LiFePO}_4$ battery. 
