\thispagestyle{plain}
\begin{center}
	\Large
	\textbf{Voice communication system for a spacesuit simulator}
	
	\vspace{0.4cm}
	\large
	Design and performance estimation of the system with additional analysis of the usability on Mars 
	
	\vspace{0.4cm}
	\textbf{Omar Filip El Sendiouny}
	
	\vspace{0.9cm}
	\textbf{Abstract}
\end{center}

In this thesis, a voice communication system with an associated self-sufficient energy distribution system for a spacesuit simulator is designed and its performance is examined. For this purpose, the plane earth signal budget is used to show that the minimum fade margin of the voice communication system is sufficiently large to meet the range requirements and a \MATLAB simulation is developed which estimates the daily electrical energy yield of the self-sufficient energy distribution system for different mission locations on Earth. The latter is based on the angular relationships between the Sun and Earth, a model of a PV generator and a model of a $\mathrm{LiFePO_4}$ battery. The performance estimation of the voice communication system has shown that a sufficiently large fade margin could be achieved if a repeater radio infrastructure is used. Regarding the self-sufficient energy distribution system, the developed \MATLAB simulation showed that it can be used to supply the repeater radio infrastructre for different mission locations on Earth when sufficient solar radiation is available. In addition, this thesis examines how the designed system must be adapted so that it can be used on the surface of Mars. Investigations showed, that the Martian Ionosphere can be used as a reflector for electromagnetic waves in the very high frequency band during the day. Thus allowing global communication. Another important finding of this investigation was that the temperature dependence of the electrical devices involved must not be neglected if such a system is to be planned for Mars. 